\setlength{\LTpost}{0mm}
\begin{longtable}{llrlrl}
\caption*{
{\large Comparison of significant immune gene responses} \\ 
{\small Controlled laboratory infections vs. field-caught mice infected with \emph{Eimeria} spp.}
} \\ 
\toprule
 &  & \multicolumn{2}{c}{Controlled laboratory infections} & \multicolumn{2}{c}{Field-caught mice} \\ 
\cmidrule(lr){3-4} \cmidrule(lr){5-6}
Gene & Species & Estimate & P-value & Estimate & P-value \\ 
\midrule\addlinespace[2.5pt]
\textbf{CXCL9} & E. falciformis & $2.08$ & < 0.001 & $3.37$ & < 0.001 \\ 
\textbf{CXCL9} & E. ferrisi & $2.88$ & < 0.001 & $1.63$ & = 0.008 \\ 
\textbf{IDO1} & E. ferrisi & $2.50$ & < 0.001 & — & — \\ 
\textbf{IFNy} & E. falciformis & $1.65$ & = 0.01 & — & — \\ 
\textbf{IFNy} & E. ferrisi & $2.02$ & = 0.003 & — & — \\ 
\textbf{NCR1} & E. falciformis & $-1.94$ & = 0.01 & — & — \\ 
\textbf{PRF1} & E. falciformis & — & — & $2.08$ & = 0.02 \\ 
\textbf{TICAM1} & E. falciformis & $-2.25$ & = 0.002 & $2.12$ & = 0.04 \\ 
\textbf{TNF} & E. ferrisi & $1.86$ & = 0.02 & — & — \\ 
\bottomrule
\end{longtable}
\begin{minipage}{\linewidth}
Only genes with p \textless{} 0.05 in at least one condition shown. Full results in Supplementary Tables S1-S2. Infections with \emph{Eimeria} spp.\\
\end{minipage}

