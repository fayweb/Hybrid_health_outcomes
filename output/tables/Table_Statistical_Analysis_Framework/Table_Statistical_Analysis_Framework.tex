\setlength{\LTpost}{0mm}
\begin{longtable}{>{\raggedright\arraybackslash}p{120px}>{\raggedright\arraybackslash}p{200px}>{\raggedright\arraybackslash}p{80px}>{\raggedright\arraybackslash}p{250px}>{\raggedright\arraybackslash}p{200px}>{\raggedright\arraybackslash}p{80px}>{\raggedright\arraybackslash}p{120px}}
\caption*{
{\large }
} \\ 
\toprule
Analysis Phase & Research Question & Model ID & Statistical Method & Key Finding & Sample Size & Performance Metric\textsuperscript{\textit{1}} \\ 
\midrule\addlinespace[2.5pt]
\multicolumn{7}{l}{Laboratory Development} \\ 
\midrule\addlinespace[2.5pt]
Discovery & Can immune genes predict infection costs? & \cellcolor[HTML]{F8F9FA}{\textcolor[HTML]{1F77B4}{\textbf{DISC-1}}} & Linear regression (PC1, PC2 → weight loss) & Significant but modest prediction & n = 136 & \textcolor[HTML]{D62728}{\textbf{R² = 0.106***}} \\ 
Optimization & Can machine learning improve prediction? & \cellcolor[HTML]{F8F9FA}{\textcolor[HTML]{1F77B4}{\textbf{DISC-2}}} & Random forest (19 genes → weight loss) & Substantial improvement achieved & n = 136 & \textcolor[HTML]{D62728}{\textbf{R² = 0.476***}} \\ 
Validation & Is the model reliable? & \cellcolor[HTML]{F8F9FA}{\textcolor[HTML]{1F77B4}{\textbf{DISC-3}}} & Train-test cross-validation & Strong predictive accuracy & n = 95→41\textsuperscript{\textit{2}} & \textcolor[HTML]{D62728}{\textbf{r = 0.79***}} \\ 
\midrule\addlinespace[2.5pt]
\multicolumn{7}{l}{Field Translation} \\ 
\midrule\addlinespace[2.5pt]
Detection & Does the model work in wild populations? & \cellcolor[HTML]{F8F9FA}{\textcolor[HTML]{1F77B4}{\textbf{FIELD-1}}} & Predicted vs. observed infection status & Successfully detects infection & n = 305 & \textcolor[HTML]{D62728}{\textbf{+1.15\%***}} \\ 
Discrimination & Can it distinguish parasite species? & \cellcolor[HTML]{F8F9FA}{\textcolor[HTML]{1F77B4}{\textbf{FIELD-2}}} & Predicted loss by species identity & Species-specific responses & n = 169 & \textcolor[HTML]{D62728}{\textbf{E.f: +2.06\%**, E.r: +1.25\%**\textsuperscript{\textit{3}}}} \\ 
Scaling & Does it correlate with infection severity? & \cellcolor[HTML]{F8F9FA}{\textcolor[HTML]{1F77B4}{\textbf{FIELD-3}}} & Predicted loss vs. parasite load & Scales with infection intensity & n = 185 & \textcolor[HTML]{D62728}{\textbf{r = 0.233*}} \\ 
\midrule\addlinespace[2.5pt]
\multicolumn{7}{l}{Biological Validation} \\ 
\midrule\addlinespace[2.5pt]
Physiological relevance & Does it capture real health impacts? & \cellcolor[HTML]{F8F9FA}{\textcolor[HTML]{1F77B4}{\textbf{PROOF-1}}} & Predicted loss vs. body condition & Correlates with actual body weight & n = 336 & \textcolor[HTML]{D62728}{\textbf{ρ = -0.115*}} \\ 
Specificity & Is the response Eimeria-specific? & \cellcolor[HTML]{F8F9FA}{\textcolor[HTML]{1F77B4}{\textbf{PROOF-2}}} & Predicted loss vs. parasite community & Specific to Eimeria infections only & n = 305 & \textcolor[HTML]{D62728}{\textbf{p < 0.001***}} \\ 
\bottomrule
\end{longtable}
\begin{minipage}{\linewidth}
\textsuperscript{\textit{1}}Significance levels: *p < 0.05, **p < 0.01, ***p < 0.001\\
\textsuperscript{\textit{2}}Progressive sample sizes reflect train→test validation approach\\
\textsuperscript{\textit{3}}E.f: Eimeria falciformis; E.r: E. ferrisi\\
Framework progresses from basic linear prediction (R² = 0.106) through machine learning optimization (R² = 0.476) to comprehensive field validation with biological relevance.\\
\end{minipage}

