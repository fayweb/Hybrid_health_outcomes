\setlength{\LTpost}{0mm}
\begin{longtable}{llllrr}
\caption*{
{\large Eimeria detection methods in field mice} \\ 
{\small Summary of hierarchical approach for infection status and species assignment}
} \\ 
\toprule
Detection Method\textsuperscript{\textit{1}} & Variable Name & Purpose & Priority & Sample Size (n/total) & Success Rate \\ 
\midrule\addlinespace[2.5pt]
\cellcolor[HTML]{F3E5F5}{Caecal qPCR + melting curve} & \cellcolor[HTML]{F3E5F5}{MC.Eimeria} & \cellcolor[HTML]{F3E5F5}{Infection detection (presence/absence)} & \cellcolor[HTML]{F3E5F5}{Primary} & \cellcolor[HTML]{F3E5F5}{185/336} & \cellcolor[HTML]{F3E5F5}{55.1\%} \\ 
\cellcolor[HTML]{F3E5F5}{Caecal qPCR + melting curve} & \cellcolor[HTML]{F3E5F5}{eimeriaSpecies} & \cellcolor[HTML]{F3E5F5}{Species identification} & \cellcolor[HTML]{F3E5F5}{Primary} & \cellcolor[HTML]{F3E5F5}{169/336} & \cellcolor[HTML]{F3E5F5}{50.3\%} \\ 
\cellcolor[HTML]{FFF3E0}{Amplicon sequencing\textsuperscript{\textit{2}}} & \cellcolor[HTML]{FFF3E0}{amplicon\_species} & \cellcolor[HTML]{FFF3E0}{Backup species identification} & \cellcolor[HTML]{FFF3E0}{Backup} & \cellcolor[HTML]{FFF3E0}{134/336} & \cellcolor[HTML]{FFF3E0}{39.9\%} \\ 
\bottomrule
\end{longtable}
\begin{minipage}{\linewidth}
\textsuperscript{\textit{1}}Primary method: Caecal tissue qPCR with melting curve analysis for both detection and species ID\\
\textsuperscript{\textit{2}}Backup method: Used only when qPCR species identification failed\\
\end{minipage}

